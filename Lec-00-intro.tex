%!TEX root = P231_notes.tex

\section{Introduction: Why mathematical methods?}
\lecdate{lec~01}

Physics 231:~Methods of Theoretical Physics is a course for first-year physics and astronomy graduate students. It is a `crash course’ in mathematical methods necessary for graduate courses in electrodynamics, quantum mechanics, and statistical mechanics. It is a \emph{boot camp} rather than a rigorous theorem--proof mathematics class. Where possible, the emphasis is on physical intuition rather than mathematical precision. 

\subsection{Green’s functions}

Our primary goal is to solve linear differential equations:
\begin{align}
  \mathcal O f(x) = s(x) \ .
\end{align}
In this equation, $\mathcal O$ is a \emph{differential operator} that encodes some kind of physical dynamics\footnote{Say, $\mathcal O = (d/dx)^2 + 3x\,(d/dx)$. Pop quiz: is this operator \emph{linear}? The first term is squared...}, $s(x)$ is the \emph{source} of those dynamics, and $f(x)$ is the system's physical \emph{response} that we would like to determine. The solution to this equation is:
\begin{align}
  f(x) &= \mathcal O^{-1} s(x) \ .
\end{align}
Simply writing that is deeply unsatisfying! %It is as if we were asked to solve $f'(x) = 3x$ and simply wrote $f(x) = \int 3x\, dx$. The real 
In this course, we think carefully about what $\mathcal O^{-1}$ actually \emph{means} and how we can calculate it. As you may have guessed, $\mathcal O^{-1}$ is the \textbf{Green's function} for the differential operator $\mathcal O$. 

We approach problem by analogy to linear algebra, where a linear transformation\footnote{Recall that as physicists, `linear transformation' is a fancy way of saying `matrix.'} $A$ acts on a vector to give equations like
\begin{align}
  A \vec{v} = \vec{w} \ ,
\end{align}
whose solution is
\begin{align}
  \vec{v} = A^{-1} \vec{w} \ .
\end{align}
We connect the notion of a linear differential operator to a matrix in an infinite dimensional space to give a working definition of $\mathcal O^{-1}$. We then pull out a bag of tricks from complex analysis to actually solve $\mathcal O^{-1}s(x)$ given $\mathcal O$ and $s(x)$. 

\subsection{This is not what I expected from a math methods course}

This is a course in mathematical methods for \emph{physicists}. We will not solve \emph{every} class of differential equation that is likely to pop up in your research careers\footnote{That would be a course on mathematical methods for \emph{engineers}.}---that's neither feasible nor particularly enjoyable. 
%
This is also not a course in formal proofs---there are plenty of excellent textbooks for you to learn those formal proofs to your heart's content\footnote{... and as a graduate student, you should feel well equipped and encouraged to learn all of the necessary material \emph{you} need for \emph{your} research and interests---whether or not they show up in your coursework.}. 
%
The goal of this course is to weave together ideas that are not often connected explicitly in undergraduate physics courses in the United States: linear algebra, differential equations, complex analysis. These ideas are not necessarily new---in fact, I \emph{expect} you have seen many them often---but rather we will take a big view of how the interconnection of these ideas come up over and over again in our description of nature.

Do not be surprised if we only mention Bessel functions in passing. Do not think less of our efforts if we do not determine Wronksians or go beyond a single Riemann sheet. As graduate students, it is \emph{your} responsibility to be able to grab your favorite textbook to apply mathematics as needed to your research. \emph{This course} is about the larger narrative that is not often shared explicitly in those books. It is the `knack for math' that physicists are, as a culture, rather proud of. It is what tends to make us employable in Silicon Valley while simultaneously terrible at splitting the bill at a restaurant. 
 
\subsection{Green’s functions}