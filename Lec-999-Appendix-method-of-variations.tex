\section{Method of Variations}
\label{app:method:of:variations}

The method of variations is a way to solve inhomogeneous differential equations starting from a basis of solutions to the homogeneous differential equation. We specialize to the case of second-order differential operators, though the technique is valid for higher-order operators.

\subsection{First order equation for coefficients}

Consider the second-order inhomogeneous linear differential equation
\begin{align}
	\mathcal Of(x) &= a_2(x)f''(x) + a_1(x) f'(x) + a_0(x) = g(x) \ .
\end{align}
One can determine the solution to this equation from varying with respect to the solutions to the homogeneous differential equation, $u(x)$ and $v(x)$. Write the solution to the inhomogeneous equation as
\begin{align}
	f(x) &= c(x)u(x) + d(x)v(x)
\end{align}
Make the ansatz that
\begin{align}
	c'(x) u(x) + d'(x) v(x) = 0 \ .
	\label{eq:method:of:variations:ansatz}
\end{align}
Then the inhomogeneous differential equation reduces to
\begin{align}
	\mathcal Of &= a_2(x)\left(c'(x) u'(x) + d'(x) v'(x)\right) =g(x) \ .
\end{align}
This gives expressions for $c'$ and $d'$:
\begin{align}
	c'(x) &= \frac{-v(x)g(x)/a_2(x)}{u(x) v'(x) - u'(x) v(x)}
	&
	d'(x) &= \frac{u(x)g(x)/a_2(x)}{u(x) v'(x) - u'(x) u(x)} \ ,
	\label{eq:method:of:variations:first:order}
\end{align}
where one may recognize the denominators to be the Wronksian, $W = uv' - u'v$. For simplicity, write $\tilde g(x) \equiv g(x)/a_2(x)$.
% Integrating these equations then gives the solution to the inhomogeneous equation. 

\subsection{Integrating and boundary conditions}

Suppose that the inhomogeneous differential equation has boundary conditions at $x_1$ and $x_2$ given by differential operators $\mathcal B_{1,2}$:
\begin{align}
	\mathcal B_1 f(x) &= \alpha_1 f'(x_1) + \beta_1 f(x_1) = 0
	&
	\mathcal B_2 f(x) &= \alpha_2 f'(x_2) + \beta_2 f(x_2) = 0 \ .
\end{align}
These boundary conditions are, in general, first order for a second order inhomogeneous differential equation. For example, these may come from integrating a second-order equation of motion by parts, leaving a first-order operator on either boundary.
%
The ansatz \eqref{eq:method:of:variations:ansatz} simplifies these boundary conditions:
\begin{align}
	\mathcal B_i f(x) &= 
	c(x_i) \mathcal B_i u (x) 
	+
	d(x_i) \mathcal B_i v (x) 
	= 0 
	\ .
	\label{eq:method:of:variations:BC}
\end{align}
For simplicity, let us write $(\mathcal B_i u) = \alpha_i u'(x_i) + \beta_i f(x_i)$ as a constant that depends on $u$ and $u'$ evaluated at the boundary $x_i$. We define $(\mathcal B_i v)$ similarly. 

Integrate the first-order differential equations for the coefficients \eqref{eq:method:of:variations:first:order} using these boundary conditions. The simplest approach is to integrate a convenient linear combinations:
\begin{align}
	\mathcal{I}_1 = 
	\int_{x_1}^x
	dy\, c'(y)(\mathcal B_1 u) + d'(y) (\mathcal B_1 v)
	&= c(x) (\mathcal B_1 u) + d(x) (\mathcal B_1 v) \ ,
	\label{eq:method:of:variations:int:1}
\end{align}
where the lower limit vanishes because $(\mathcal B_i f) \equiv 0$. Similarly, one finds
\begin{align}
	\mathcal{I}_2 = 
	\int_{x}^{x_2}
	dy\, c'(y)(\mathcal B_2 u) + d'(y) (\mathcal B_2 v)
	&= -c(x) (\mathcal B_2 u) - d(x) (\mathcal B_2 v) \ .
	\label{eq:method:of:variations:int:2}
\end{align}
The integrals in \eqref{eq:method:of:variations:int:1} and \eqref{eq:method:of:variations:int:2} are equivalently expressed using the expressions for $c'(x)$ and $d'(x)$ in \eqref{eq:method:of:variations:first:order}:
% \begin{align}
% 	\int_{x_1}^x
% 	dy\, c'(y)(\mathcal B_1 u) + d'(y) (\mathcal B_1 v)
% 	&= 
% 	\int_{x_1}^x dy \,
% 	\frac{-v(y)
% 	% g(x)/a_2(x)
% 	\tilde g(y)
% 	}{W(y)}
% 	(\mathcal B_1 u)
% 	+ 
% 	\frac{u(y)
% 	%g(x)/a_2(x)
% 	\tilde g(y)
% 	}{W(y)}
% 	(\mathcal B_1 v) 
% 	\\
% 	\int_{x}^{x_2}
% 	dy\, c'(y)(\mathcal B_2 u) + d'(y) (\mathcal B_2 v)
% 	&=
% 	\int_{x}^{x_2}
% 	dy\, \frac{-v(y)
% 	%g(x)/a_2(x)
% 	\tilde g(y)
% 	}{W(y)} 
% 	(\mathcal B_2 u) 
% 	+ \frac{u(y)
% 	%g(x)/a_2(x)
% 	\tilde g(y)
% 	}{W(y)} 
% 	(\mathcal B_2 v)
% 	\ .
% \end{align}
\begin{align}
	\mathcal{I}_1 
	%= 
	% \int_{x_1}^x
	% dy\, c'(y)(\mathcal B_1 u) + d'(y) (\mathcal B_1 v)
	&= 
	\int_{x_1}^x dy \,
	\left[
	- (\mathcal B_1 u) v(y)
	+ (\mathcal B_1 v) u(y)
	\right]
	\frac{
	\tilde g(y)
	}{W(y)}	
	\\
	\mathcal{I}_2 
	% = 
	% \int_{x}^{x_2}
	% dy\, c'(y)(\mathcal B_2 u) + d'(y) (\mathcal B_2 v)
	&=
	\int_{x}^{x_2}
	dy\, 
	\left[  - (\mathcal B_2 u) v(y) 
			+ (\mathcal B_2 v) u(y)
	\right]
	\frac{
	\tilde g(y)
	}{W(y)} 
	\ .
\end{align}
Combining this with the right-hand sides of \eqref{eq:method:of:variations:int:1} and \eqref{eq:method:of:variations:int:2} then gives the desired integral solutions for the coefficient functions:
\begin{align}
	\left[ \left(\mathcal B_1 u\right)\left(\mathcal B_2 v\right) - \left(\mathcal B_2 u\right)\left(\mathcal B_1 v\right)  \right]
	c(x)
	&= 
	% \phantom{+}
	\left(\mathcal B_2 v\right)
	\mathcal{I}_1 
	% \int_{x_1}^x dy \,
	% \left[
	% - (\mathcal B_1 u) v(y)
	% + (\mathcal B_1 v) u(y)
	% \right]
	% \frac{
	% \tilde g(y)
	% }{W(y)}	
	% \\
	% &\phantom{=}
	+
	\left(\mathcal B_1 v\right)
	% \int_{x}^{x_2}
	% dy\, 
	% \left[  - (\mathcal B_2 u) v(y) 
	% 		+ (\mathcal B_2 v) u(y)
	% \right]
	% \frac{
	% \tilde g(y)
	% }{W(y)} 
	\mathcal{I}_2 
	\\
	\left[ 
		\left(\mathcal B_2 u\right)\left(\mathcal B_1 v\right) 
		- 
		\left(\mathcal B_1 u\right)\left(\mathcal B_2 v\right)  
	\right]
	d(x)
	&= 
	\left(\mathcal B_2 u\right)
	\mathcal{I}_1 
	+ 
	\left(\mathcal B_1 u\right)
	\mathcal{I}_2 
	\ .
\end{align}
% Similarly for the $d(x)$ coefficient:
% \begin{align}
% 	\left[ 
% 		\left(\mathcal B_2 u\right)\left(\mathcal B_1 v\right) 
% 		- 
% 		\left(\mathcal B_1 u\right)\left(\mathcal B_2 v\right)  
% 	\right]
% 	d(x)
% 	&= 
% 	\phantom{+}
% 	\left(\mathcal B_2 u\right)
% 	\int_{x_1}^x dy \,
% 	\left[
% 	- (\mathcal B_1 u) v(y)
% 	+ (\mathcal B_1 v) u(y)
% 	\right]
% 	\frac{
% 	\tilde g(y)
% 	}{W(y)}	
% 	\\
% 	&\phantom{=}+
% 	\left(\mathcal B_1 u\right)
% 	\int_{x}^{x_2}
% 	dy\, 
% 	\left[  - (\mathcal B_2 u) v(y) 
% 			+ (\mathcal B_2 v) u(y)
% 	\right]
% 	\frac{
% 	\tilde g(y)
% 	}{W(y)} 
% 	\ .
% \end{align}
Define the left-hand side prefactors:
\begin{align}
	(\text{den}.) &\equiv
	\left(\mathcal B_1 u\right)\left(\mathcal B_2 v\right) - \left(\mathcal B_2 u\right)\left(\mathcal B_1 v\right)  \ ,
\end{align}
so that we have 
\begin{align}
	f(x) &= 
	\frac{\left(\mathcal B_2 v\right)
		\mathcal{I}_1 
		+
		\left(\mathcal B_1 v\right)
		\mathcal{I}_2}{(\text{den.})}
	u(x)
	-
	\frac{\left(\mathcal B_2 u\right)
		\mathcal{I}_1 
		+
		\left(\mathcal B_1 u\right)
		\mathcal{I}_2}{(\text{den.})}
	v(x)
\end{align}
Collecting the terms by the integration range gives:
\begin{align}
	(\text{den.})f(x)&=
	\int_{x_1}^x dy \frac{\tilde g(y)}{W(y)}
	\left[
		-(\mathcal B_1u)(\mathcal B_2u) v(x)v(y)
		+(\mathcal B_1v)(\mathcal B_2v) u(x)u(y)
		\right.
	\\
	&\phantom{=\int_{x_1}^x dy \frac{\tilde g(y)}{W(y)}[}
		\left.
		-(\mathcal B_1u)(\mathcal B_2v) u(x)v(y)
		+(\mathcal B_1v)(\mathcal B_2u) v(x)u(y)
	\right] + \cdots
	\\
	&=
	\int_{x_1}^x dy \frac{\tilde g(y)}{W(y)}
	\left[
		(\mathcal B_1u)v(y) - (\mathcal B_1v)u(y)
	\right] 
	\left[
		(\mathcal B_2u)v(x) - (\mathcal B_2v)u(x)
	\right] + \cdots \ .
\end{align}
The $\cdots$ represent the $\mathcal I_2$ terms. Performing the same grouping for the $\mathcal I_2$ terms then gives:
\begin{align}
	(\text{den.})f(x)&=
	\int_{x_1}^x dy \frac{\tilde g(y)}{W(y)}
	\left[
		(\mathcal B_1u)v(y) - (\mathcal B_1v)u(y)
	\right] 
	\left[
		(\mathcal B_2u)v(x) - (\mathcal B_2v)u(x)
	\right] + \\
	&\phantom{=}
	\int_{x}^{x_2} dy \frac{\tilde g(y)}{W(y)}
	\left[
		(\mathcal B_1v)u(x) - (\mathcal B_1u)v(x)
	\right] 
	\left[
		(\mathcal B_2v)u(y) - (\mathcal B_2u)v(y)
	\right] \ .
\end{align}
These two terms can be combined rather nicely:
\begin{align}
	f(x) &=
	\frac{1}{(\text{den.})}
	\int_{x_1}^{x_2} dy \,
	\frac{\tilde g(y)}{W(y)}
	\left[
		(\mathcal B_1u)v(x_<) - (\mathcal B_1v)u(x_<)
	\right] 
	\left[
		(\mathcal B_2u)v(x_>) - (\mathcal B_2v)u(x_>)
	\right]
\end{align}
where we introduce the convenient notation that
\begin{align}
	x_< &= \min(x,y)
	&
	x_> &= \max(x,y) \ .
\end{align}
